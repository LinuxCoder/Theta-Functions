\documentclass{article}
\usepackage[T1]{fontenc}
\usepackage[utf8]{inputenc}
\usepackage{amsmath,amssymb, amsthm}
\usepackage[russian]{babel}
\usepackage[left=1.5cm, right=2cm]{geometry}
\usepackage{tabularx}


\newcommand{\ZZ}{\mathbb{Z}}
\newcommand{\CC}{\mathbb{C}}
\newcommand{\NN}{\mathbb{N}}
\newcommand{\RR}{\mathbb{R}}
\newcommand{\QQ}{\mathbb{Q}}



\newtheoremstyle{break}% name
{}%         Space above, empty = `usual value'
{}%         Space below
{\itshape}% Body font
{}%         Indent amount (empty = no indent, \parindent = para indent)
{\bfseries}% Thm head font
{.}%        Punctuation after thm head
{\newline}% Space after thm head: \newline = linebreak
{}%         Thm head spec

\theoremstyle{break}


\newtheorem{theorem}{Теорема}[section]
\newtheorem{definition}[theorem]{Определение}
\newtheorem{claim}{Утверждение}[section]
\newtheorem{corollary}{Следствие}[section] % Use theorem counter as `parent`
\newtheorem{lemma}{Лемма}[section]

\newcommand{\ModkN}[2]{\text{Mod}_{#1}^{(#2)}}
\newcommand{\ModN}[1]{\text{Mod}^{(#1)}}


\usepackage{enumerate}
\usepackage[shortlabels]{enumitem}

%opening
\title{}
\author{}

\begin{document}

%%
%% Title page
%%
\begin{center}
	{\scshape Федеральное государственное автономное\\
		образовательное учреждение высшего образования\\
		<<Национальный исследовательский университет\\
		<<Высшая школа экономики>>\\[1ex]
		Факультет математики\par}
	
	\par\vfill
	
	\textbf{\large Минасян Левон Лерментович}
	
	\vspace{1.5cm}
	
	{\Large\bfseries
		О некоторых решеточных тета-рядах
		\par}
	
	\vspace{1.5cm}
	
	Курсовая работа студента 3 курса\\[1ex]
	образовательной программы бакалавриата <<Математика>>
	\par\vfill
	\noindent\hspace{0.52\textwidth}\parbox[t]{0.48\textwidth}{%
		Научный руководитель:\\[3pt]
		\textbf{Дунин-Барковский Петр Игоревич}\\[2ex]
	}%
	\par\vfill
	Москва 2023
\end{center}
\thispagestyle{empty}
\pagebreak
%%
%% ===========================================================================
%%

\section{Введение}
В этой статье мы описываем тета-функции и тета-ряды различных решеток, а также 
устанавливаем их модулярность относительно различных подгрупп модулярной группы
$SL_2(\ZZ)$.
\subsection{Историческая справка}

\subsection{Почему интересно}

\subsection{Результаты}

\section{Тета-функции}
В данной главе мы определим тета-функции Якоби, а также тета-функции с рациональными характеристиками. Получим выражения для рядов тета-функций с полуцелыми характеристиками.
\begin{definition}
	Тета-функцией (Якоби) $\theta(z, \tau)$ называется функция
	
	\begin{equation}
	\theta(z,\tau)=\sum_{n \in \ZZ} \exp(2 \pi inz + \pi i n^2 \tau)
	\end{equation}
	
	
	где $z \in \CC$ и $\tau \in H=\{\Im \tau > 0\}$ - верхняя полуплоскость.
\end{definition}

Часто бывает удобно рассмотреть $z=0$ и заменить аргумент $\tau$ на $q=e^{i \pi \tau}$. 
В таких случаях будем писать $\theta(q)$ вместо $\theta(0, \tau)$.

\begin{definition}
Тета-функцией $\theta_{a, b}$ с рациональными характеристиками $a, b \in  \frac{1}{l}\ZZ, l \in \NN$ называется
\begin{equation}
	\theta_{a, b}(z, \tau)=T_aS_b\theta(z, \tau)
\end{equation}

где $T_a, S_b$ суть преобразования сдвига аргумента $z$: для $f=f(z): \CC \to \CC$ при фиксированном $\tau \in H$
\end{definition}
\begin{equation}
	S_b f(z)=f(z+b)
\end{equation}

\begin{equation}
	T_a f(z)=\exp{(\pi i a^2 \tau + 2 \pi i a z)}f(z + a\tau)
\end{equation}

Легко показать, что $S_b$ и $T_a$ задают однопараметрические семейства преобразований $S_{b_1 + b_2}=S_{b_1}S_{b_2}$ и $T_{a_1+a_2}=T_{a_1}T_{a_2}$. 
Что не менее важно, они не коммутируют между собой: 

\begin{equation}
T_a S_b f(z)=T_a f(z + b) = \exp{(\pi i a^2 \tau + 2 \pi i a (z + b))} f(z + a\tau + b) = \exp{(2 \pi i a b)} S_b T_a f(z)
\end{equation}

т.е. $T_a S_b = \exp(2 \pi i a b) S_b T_a$

\begin{definition}
	Тета-функциями с полуцелыми характеристиками называются 4 функции
	\begin{equation}
		\begin{split}
			\theta_1(q)=\theta_{1/2, 1/2}(0, \tau) \\
			\theta_2(q)=\theta_{1/2, 0}(0, \tau) \\
			\theta_3(q)=\theta_{0, 0}(0, \tau) \\
			\theta_4(q)=\theta_{0, 1/2}(0, \tau) \\
		\end{split}
	\end{equation}
	где, как и ранее, $q=e^{\pi i \tau}$.
\end{definition}

\begin{theorem}
Имеют место следующие равенства

\begin{gather*}
	\theta_1(q)=-\sum_{n \in \ZZ} (-1)^{n-1/2}q^{(n+1/2)^2} \\ 
	\theta_2(q)=\sum_{n \in \ZZ} q^{(n+1/2)^2} \\
	\theta_3(q)=\sum_{n \in \ZZ} q^{n^2}	\\
	\theta_4(q)=\sum_{n \in \ZZ} {(-1)^n q^{n^2}}
\end{gather*}

где, как и ранее, $q=e^{\pi i \tau}$

\end{theorem}

\begin{proof}
	
\begin{equation*}
	\begin{split}
\theta_{4}(q) 
= \theta_{0, 1/2}(0, \tau) = 
\sum_{n \in \ZZ} \exp{(2 \pi i n(0 + \frac{1}{2}) + \pi i n^2 \tau)} = \\
\sum_{n \in \ZZ} (-1)^n \exp{(\pi i n^2 \tau)} = 
\sum_{n \in \ZZ} (-1)^n q^{n^2}
	\end{split}
\end{equation*}

\begin{equation*}
	\begin{split}
\theta_{2}(q) 
= \theta_{1/2, 0}(0, \tau) = 
\exp{(\pi i (1/2)^2 \tau + 2 \pi i (1/2) \cdot 0)}
\sum_{n \in \ZZ} \exp{(2 \pi i n(0 + \frac{1}{2}\tau) + \pi i n^2 \tau)} = \\
\sum_{n \in \ZZ} \exp{(\pi i (n^2 + n + 1/4) \tau)} = 
\sum_{n \in \ZZ} q ^ {(n+1/2)^2}
	\end{split}
\end{equation*}

\begin{equation*}
	\begin{split}
		\theta_{1}(q)
=\theta_{1/2, 1/2}(0, \tau) = 
\exp{(\pi i (1/2)^2 \tau + 2 \pi i (1/2) \cdot (0 + 1/2))}
\sum_{n \in \ZZ} \exp{(2 \pi i n(0 + 1/2 + \frac{1}{2}\tau) + \pi i n^2 \tau)} = \\
\sum_{n \in \ZZ} \exp{(\pi i (n + 1/2))} \exp{(\pi i (n^2 + n + 1/4) \tau)} = 
\sum_{n \in \ZZ} (-1)^{n+1/2} q ^ {(n+1/2)^2}
	\end{split}
\end{equation*}

\end{proof}

Отныне для удобства $\theta_{0, 0}, \theta_{0, 1/2}, \theta_{1/2, 0}, \theta_{1/2, 1/2}$
будем обозначать как $\theta_{00}, \theta_{01}, \theta_{10}, \theta_{11}$ соответственно.


\section{Модулярные формы}

\subsection{Определения и базовые свойства}
Определим понятие модулярных форм и докажем базовые утверждения.

Рассмотрим набор проекций
\begin{equation}
		\gamma_N: SL_2(\ZZ) \to SL_2 (\ZZ / n \ZZ)
\end{equation}


Обозначим $\Gamma_N := \ker \gamma_N$. Ясно, что ядра $\Gamma_N$ таких отображений состоят из матриц $\begin{bmatrix} a & b \\ c & d \end{bmatrix}$, где $a, b, c, d \in \ZZ$ и $a \equiv d \equiv 1(\mod N)$, $b \equiv c \equiv 0(\mod N)$.


\begin{definition}
	Модулярной формой веса $k$ уровня $N$ называется голоморфная функция $f: H \to \CC$, удовлетворяющая следующим условиям:
	
	\begin{enumerate}[start=1, label={(\bfseries M\arabic*):}]
		\item $\forall \begin{bmatrix}
			a & b \\ c & d
		\end{bmatrix} \in \Gamma_N$ верно:

	\begin{equation}
			f \Big( \frac{a\tau + b}{c\tau + d} \Big) = (c\tau + d)^k f(\tau)
	\end{equation}

	
	
		\item $\exists c, d >0$ такие, что $|f(\tau)| \le c$
				при $\Im \tau \ge d$
	
		\item $\forall s \in \QQ \exists c_s, d_s > 0:$
		\begin{equation}
			|f(\tau)| \le \frac{c_s}{|\tau - s|^k}
		\end{equation}
			
		при $|\tau - (s + i \cdot d_s)| \le d_s$
	\end{enumerate}
\end{definition}



Модулярные формы веса $k$ уровня $N$ будем обозначать $\ModkN{k}{N}$. 

Введем обозначение $e_\gamma=(c \tau + d)^k$ для $ \gamma = \begin{bmatrix}
	a & b \\ c & d \end{bmatrix}\in SL_2(\ZZ)$. Вес $k$ будет определяться из контекста.

\begin{claim}
	$\ModkN{k}{N}$ образуют векторное пространство над $\CC$.
\end{claim}
\begin{proof}
Пусть $f, g \in  \ModkN{k}{N}$. Тогда $f(\gamma \tau) = e_{\gamma}(\tau) f(\tau)$
и $g(\gamma \tau) = e_{\gamma}(\tau) g(\tau)$. Значит,
$$
(f + g)(\gamma \tau) = f(\gamma(\tau)) + g(\gamma(\tau)) 
= e_\gamma(\tau)f(\tau) + e_\gamma(\tau)g(\tau)
= e_\gamma(\tau)(f(\tau) + g(\tau))
$$

Проверим $(M2)$: 
$$
|f(\tau) + g(\tau)| \le C_f + C_g \quad \text{при} \quad 
\Im \tau \ge \max(D_f, D_g)
$$
где $(C_f, D_f), (C_g, D_g)$ константы из $(M2)$ для $f, g$.

Проверим $(M3)$. Фиксируем произвольный $s \in \QQ$. Тогда для некоторых 
$c_s^f, d_s^f, c_s^g, d_s^g > 0$:
\begin{gather*}
	|f(\tau)| \le \frac{c_s^f}{|\tau - s|^k} \quad \text{при} \quad 
		|\tau - s - i \cdot d_s^f| \le d_s^f \\	
	|g(\tau)| \le \frac{c_s^g}{|\tau - s|^k} \quad \text{при} \quad 
		|\tau - s - i \cdot d_s^g| \le d_s^g \\
\end{gather*}

Обозначим $c_s:=c_s^f + c_s^g, d_s:=\min(d_s^f, d_s^g)$. Тогда
$$
	|f(\tau) + g(\tau)| \le \frac{c_s}{|\tau - s|^k} \quad \text{при} \quad
	|\tau - s - i \cdot d_s| \le d_s
$$


Для $\lambda \in \CC$ все три условия $(M1), (M2), (M3)$, очевидно, выполняются.


\end{proof}

\begin{theorem}
	$\ModN{N}$ образует градуированную алгебру:
	$$
	\ModN{N} \stackrel{def}{=} \bigoplus_{k \ge 0} \ModkN{k}{N}
	$$
\end{theorem}

\begin{proof}
	Пусть $f \in \ModkN{k}{N}, g \in \ModkN{l}{N}$. Достаточно показать, что 
	$fg \in \ModkN{k+l}{N}$. 
	
	Условия $(M1), (M2), (M3)$ для $f \cdot g$ проверяются аналогично $f+g$ из утверждения выше.
\end{proof}


\begin{definition}
	Голоморфная функция $f: H \to \CC$ называется модулярной формой веса $k$ относительно подгруппы $\Gamma \subset SL_2(\ZZ)$, если она удовлетворяет условию $(M1)$ для всех элементов $\gamma \in \Gamma$.
\end{definition}


\begin{claim}
	$$
	e_{\gamma_1 \gamma_2}(\tau) = e_{\gamma_1}(\gamma_2(\tau)) e_{\gamma_2}(\tau)
	$$
\end{claim}
\begin{proof}
	Достаточно разделить и умножить левую часть на $e_{\gamma_2(\tau)}$
\end{proof}

\begin{claim}
	Если функция $f$ удовлетворяет условию модулярности $(M1)$ для $\gamma_1, \gamma_2 \in SL_2\big(Z)$, то она удовлетворяет $(M1)$ и для $\gamma_1 \gamma_2$.
\end{claim}

\begin{proof}
	Запишем условие $(M1)$ для $\gamma_1$ и $\gamma_2$:
	$$
	f(\gamma_1(\tau)) = e_{\gamma_1}(\tau) f(\tau)
	$$
	$$
	f(\gamma_2(\tau)) = e_{\gamma_2}(\tau) f(\tau)
	$$
	Сразу следует нужное равенство: 
	$$
	f(\gamma_1(\gamma_2(\tau))) = e_{\gamma_1}(\gamma_2(\tau)) f(\gamma_2(\tau))
	= e_{\gamma_1}(\gamma_2(\tau)) e_{\gamma_2}(\tau) f(\tau) 
	= e_{\gamma_1 \gamma_2}(\tau) f(\tau)
	$$
\end{proof}


\begin{corollary}
	Функция $f$ модулярна относительно $\Gamma \subset SL_2(\ZZ)$ тогда и только тогда, когда она удовлетворяет $(M1)$ для генераторов $\Gamma$.
\end{corollary}

\subsection{Действие $SL_2(\ZZ)$ на тета-функции}
	Выведем вспомогательное функциональное уравнение, которое в дальнейшем поможет нам 
	установить модулярность тета-функций.
\begin{theorem}
	Для тета-функции $\theta(z, \tau)$ и любого набора целых $a, b, c, d \in \ZZ$, $ad - bc=1$, $ab$ и $cd$ четны,  верны следуещие функциональные равенства:
	
	\begin{enumerate}[start=1, label={(\bfseries F\arabic*):}]
		\item 
		$$
		\theta \Big( \frac{z}{c \tau + d}, \frac{a \tau + b}{c \tau + d} \Big) = 
		\zeta (c \tau + d)^{1/2} \exp{(\frac{\pi i c z^2}{c\tau + d})} \theta(z, \tau)
		$$
	\end{enumerate}
	
	где $\zeta^8=1$ и $\zeta$ определяется в зависимости от четности $c$, $d$:
	\begin{enumerate}
		\item c четно, d нечетно
		$$
		\zeta = i^{(d - 1)/2} \cdot \Big( \frac{c}{|d|} \Big)
		$$
		\item c нечетно, d четно
		$$
		\zeta = \exp{(-\pi i c / 4)} \cdot \Big( \frac{d}{c} \Big)
		$$
		
		где $\Big( \frac{x}{y} \Big)$ -- символ Якоби для x, y.
	\end{enumerate}
\end{theorem}
\begin{proof}
	При $c=0$ равенство, очевидно, выполняется: $d=\pm 1, a=\pm 1$. 
	Отныне будем считать $c > 0$.
	
	Рассмотрим функцию
	$$
		\Psi(y, \tau) = \exp(\pi i c (c\tau + d) y^2) \theta((c\tau + d)y, \tau)
	$$
	Она будет периодичной относительно сдвигов $y \mapsto y + 1$ и квазипериодичной относительно сдвига $y \mapsto y + (a\tau+b)/(c\tau + d)$:
	$$
		\Psi(y + \frac{a\tau+b}{c\tau+d}) 
		= \exp(-\pi i \frac{a\tau+b}{c\tau+d} - 2\pi i y) \cdot \Psi(y, \tau)
	$$
	$$
	\Psi(y, \tau) = \phi(\tau)\theta(y, (a\tau + b)/(c\tau+d))
	$$
	где $\phi(\tau)$ -- некоторая неизвестная функция.
	Оба эти равенства показать нетрудно, см. \cite{mumford} \S 7.
	$$
	\theta(z, \tau)=\phi(\tau)\exp(-\pi i cz^2/(c\tau+d)) 
	\theta(z/(c\tau+d), ())
	$$
	
	Проинтегрируем обе части нижнего равенства по $y$ на отрезке $[0, 1]$:
	$$
	\int_{[0, 1]} \Psi(y, \tau) dy 
	= \phi(\tau) \cdot \int_{[0, 1]} \theta(y, (a\tau+b)/(c\tau + d))
	= \phi(\tau)
	$$
	\begin{gather*}
	\phi(\tau)
	= \int_{[0, 1]} \exp(\pi i c (c\tau + d) y^2) \theta((c\tau+d)y, \tau) dy \\
	= \sum_{n \in \ZZ} \exp(-\pi i n^2 d/c)  
		\int_{[0, 1]} \exp(\pi i (cy + n)^2(\tau+d/c))dy \\
	= \sum_{n = 1...c} \exp(-\pi i n^2 d/c) 
		\int_{\RR} \exp(\pi i c^2 y^2 (\tau + d/c)) dy
	\end{gather*}

\begin{gather*}
	\int_{\RR}(\pi i c^2 y^2 (\tau + d/c)) dy = |\tau=it-d/c| \\
	=\int_{\RR}\exp(-\pi c^2 y^2 t) dy = |u=ct^{1/2}y|
	= \frac{1}{ct^{1/2}} \int_{\RR} \exp(-\pi u^2) du = \frac{1}{ct^{1/2}}
\end{gather*}

	Значит, 
\begin{gather*}
	\phi(\tau)
	= \frac{1}{c \cdot ((\tau+d/c)/i)^{1/2}} 
	\sum_{n=1..c} \exp(-\pi i n^2 d/c) \\
	= \frac{1}{c\tau+d} \zeta
\end{gather*}

	где $\zeta^8=1$
	
	Индукцией по |c| нетрудно показать, что 
	$\zeta$ равен в точности тому, о чем говорится в утверждении, см. \cite{mumford} \S 7.
\end{proof}


\begin{corollary}
	\begin{tabularx}{\textwidth}{@{}X|X@{}}
		$$\theta_{00}(z, \tau + 1) = \theta_{01}(z, \tau)$$
		$$\theta_{01}(z, \tau + 1) = \theta_{00}(z, \tau)$$
		$$\theta_{10}(z, \tau + 1) = \exp(\pi i/4)\theta_{10}(z, \tau)$$
		$$\theta_{11}(z, \tau + 1) = \exp(\pi i/4)\theta_{11}(z, \tau)$$
		&
		$$\theta_{00}(\frac{z}{\tau}, -\frac{1}{\tau})
			=(-i\tau)^{1/2}\exp(\pi i z^2 / \tau) \theta_{00}(z, \tau)$$
		$$\theta_{01}(\frac{z}{\tau}, -\frac{1}{\tau})
			=(-i\tau)^{1/2}\exp(\pi i z^2 / \tau) \theta_{10}(z, \tau)$$
		$$\theta_{10}(\frac{z}{\tau}, -\frac{1}{\tau})
			=(-i\tau)^{1/2}\exp(\pi i z^2 / \tau) \theta_{01}(z, \tau)$$
		$$\theta_{11}(\frac{z}{\tau}, -\frac{1}{\tau})
			=-(-i\tau)^{1/2}\exp(\pi i z^2 / \tau) \theta_{11}(z, \tau)$$
	\end{tabularx}
\end{corollary}
\begin{proof}
	Формулы слева получаются прямой подстановкой в соответствующий ряд. Формулы справа получаются подстановкой в уравнение (F1).
\end{proof}

\subsection{Другое действие $SL_2(\ZZ)$ на модулярные формы}
До этого мы рассматривали действие $SL_2(\ZZ)$ на голоморфных функциях $f: H \to \CC$
 в виде дробно-линейного преобразования аргумента. Оно "портило" $f$ в том смысле, что если $f(\tau)$ была модулярной формой, то 
 $f(\gamma(\tau)) = (c\tau + d)^kf(\tau)$, вообще говоря,
  модулярной формой уже не являлась.
 
 Для произвольного $\gamma \in SL_2(\ZZ) / \Gamma_N$, $f \in \ModkN{k}{N}$ рассмотрим
 $$
 	[f(\tau)]^\gamma = e_{\gamma}^{-1}(\tau) f(\gamma\tau)
 $$
 
\begin{claim}
	$[f(\tau)]^\gamma$ будет модулярной формой того же веса и уровня, что и 
	$f(\tau) \in \ModkN{k}{N}$.
\end{claim}
\begin{proof}
	$$
	[f(\alpha \tau)]^\gamma = \frac{f(\gamma\alpha\tau)}{e_{\gamma}(\alpha \tau)}
	= \frac{f(\gamma\alpha\tau)}{e_{\gamma\alpha}(\tau)} \cdot e_\alpha(\tau)
	$$
	но $\gamma\alpha \in \Gamma_N$, поэтому
	$$
	f(\gamma\alpha \tau) = e_{\gamma \alpha}(\tau) f(\tau)
	$$
\end{proof}


\begin{claim}
	\begin{tabularx}{\textwidth}{@{}X|X@{}}
		$$[\theta_{00}^2(0, \tau)]^\alpha=\theta_{01}^2(0, \tau)$$
		$$[\theta_{00}^2(0, \tau)]^\alpha=\theta_{00}^2(0, \tau)$$
		$$[\theta_{00}^2(0, \tau)]^\alpha=i\theta_{10}^2(0, \tau)$$
		&
		$$[\theta_{00}^2(0, \tau)]^\beta=-i\theta_{00}^2(0, \tau)$$
		$$[\theta_{00}^2(0, \tau)]^\beta=-i\theta_{10}^2(0, \tau)$$
		$$[\theta_{00}^2(0, \tau)]^\beta=-i\theta_{01}^2(0, \tau)$$
	\end{tabularx}
\end{claim}
\begin{proof}
	Достаточно подставить $z=0$ в нужные уравнения утверждения 6.5.
\end{proof}

\subsection{Как строить модулярные формы с помощью тета-функций} 
Установим факт модулярности тета-функций, а также покажем, как построить функцию, модулярную относительно всей $SL_2(\ZZ)$.
\begin{claim}
	Квадраты $\theta_{00}(z=0, \tau)^2, \theta_{01}(z=0, \tau)^2, \theta_{10}(z=0, \tau)^2$ тета-констант являются модулярными формами веса $1$ уровня $4$.
\end{claim}
\begin{proof}
	Из равенства $(F1)$ для тета-константы $\theta_{00}(0, \tau)$ следует:
	$$
		\theta_{00}\Big(0, \frac{a \tau + b}{c \tau + d}\Big)^2 =
		\zeta^2 (c\tau + d) \theta_{00}(0, \tau)^2
	$$
	
	и в силу $\begin{bmatrix}
		a & b \\ c & d
	\end{bmatrix} \in \Gamma_4$ можно заключить $\zeta=\pm 1$.

	
	Теперь, в силу равенств из утверждения выше, $\theta_{01}(0, \tau), \theta_{10}(0, \tau)$
	 являются модулярными формами того же веса и уровня, что и $\theta_{00}(0, \tau)$.
	 
	Более того, пространство 
	$\langle 
	\theta^2_{00}(0, \tau), \theta^2_{01}(0, \tau), \theta^2_{10}(0, \tau) \rangle
	\subset \ModkN{1}{4}
	$
	$SL_2(\ZZ)$-инвариантно, поэтому
	для проверки условий ограниченности $(M2),(M3)$ в $s \in \QQ \cup \infty$ достаточно ограниченности всех трех функций
	вблизи $s = \infty$, т.к. всегда можно найти подходящее преобразование 
	$\gamma \in SL_2(\ZZ)$ т.ч. $\gamma(s)=\infty$.
	
	\begin{gather*}
			|\theta_{00}(0, \tau)| = |\sum_{n \in \ZZ} \exp(\pi i n^2 \tau)|
			\le \sum_{n \in \ZZ} \exp(-\pi i n^2 \cdot \Im \tau) \\
			= 1 + 2 \sum_{n \in \NN} \exp(-\pi i \cdot \Im \tau) ^ {n^2}. \\
			\sum_{n \in \NN} \exp(-\pi i \cdot \Im \tau) ^ {n^2} \le \frac{t}{1-t} = O(t) \quad \text{при} \quad \Im \tau \to \infty,
	\end{gather*}

	где $t=\exp(-\pi i \cdot \Im \tau) \to 0$ при $\Im \tau \to \infty$.

	Точно так же, как и для $\theta_{00}$ получаем ограниченность при $\Im \tau \to \infty$:
	$$
	\theta_{01}(0, \tau) = 1 + O(\exp(-\pi \Im \tau))
	$$
	$$
		\theta_{10}(0, \tau) = O(\exp(-\pi \Im \tau))
	$$
\end{proof}

\begin{claim}
	$\theta_{00}^8(0, \tau) + \theta_{01}^8(0, \tau) + \theta_{10}^8(0, \tau)$ является модулярной формой веса 4 относительно всей модулярной группы $SL_2(\ZZ)$.
\end{claim}

\begin{proof}
	
	По-прежнему, $\alpha=\begin{bmatrix}
		1 & 1 \\ 0 & 1
	\end{bmatrix}, \beta=\begin{bmatrix}
		0 & -1 \\ 1 & 0
	\end{bmatrix}$ -- генераторы $SL_2(\ZZ)$.

	Из утверждения выше: 
	$\theta_{00}^8(0, \alpha\tau)=\theta_{01}^8(0, \tau)$, 
	$\theta_{01}^8(0, \alpha\tau)=\theta_{00}^8(0, \tau)$,
	$\theta_{00}^8(0, \alpha\tau)=\theta_{01}^8(0, \tau)$, 
	а также:
	$$
	\theta_{00}^8(0, \beta \tau)=
	\Big( \theta_{00}^2(0, \beta \tau) \Big)^4=
	\Big( -i\tau\theta_{00}^2(0, \tau) \Big)^4=
	\tau^4\theta_{00}^8(0, \tau).
	$$
	
	Аналогично $\theta_{01}^8(0, \beta\tau)=\tau^4\theta_{10}^8(0, \tau)$ и
	$\theta_{00}^8(0, \beta \tau)=\tau^4\theta_{10}^8(0, \tau)$.

	Значит,
	$$
	\theta_{00}^8(0, \beta \tau) 
	+ \theta_{01}^8(0, \beta \tau)
	+ \theta_{10}^8(0, \beta \tau)=
	\tau^4(\theta_{00}^8(0, \tau) + \theta_{01}^8(0, \tau) + \theta_{10}^8(0, \tau))
	$$
	
\end{proof}

\section{Решеточные тета-ряды}
Определим решеточные тета-ряды и для некоторых решеток выразим их тета-ряды через 
тета-функции с полуцелыми характеристиками.
\begin{definition}
	Решеточным тета-рядом для целочисленной решетки $\Lambda \subset \RR^k$ называется функция
	$$
		\theta_{\Lambda}(q)=\sum_{x \in \Lambda} q^{x \cdot x}
	$$
	где $x \cdot y$ означает скалярное произведение в смысле $\sum_{i} x_i y_i$. 
	Мы будем обозначать $N(x) = x \cdot x$.
\end{definition}

Например, для решетки $\Lambda=\ZZ$ соответствующий тета-ряд $\theta_\ZZ$ будет равен

$$
	\theta_{\ZZ}(q) = \sum_{x \in \ZZ} q^{x \cdot x} = 
	\sum_{x \in \ZZ} \exp(\pi i x^2 \tau) = \theta_{00}(0, \tau)
$$


\begin{claim}
	$\Lambda \subset \RR^k, \Omega \subset \RR^l$. Тогда тета-ряд решетки 
	$\Lambda \oplus \Omega \subset \RR^{k + l}$ есть произведение тета-рядов
	решеток $\Lambda$ и $\Omega$.
\end{claim}
\begin{proof}
	Заметим, что если $x = (x_1, x_2)$, где $x_1 \in \Lambda, x_2 \in \Omega$, то 
	$x \cdot x = x_1 \cdot x_1 + x_2 \cdot x_2$. Тогда 
	$$
		\sum_{x \in \Lambda \oplus \Omega} q ^ {x \cdot x}
		= \sum_{x_1 \in \Omega, x_2 \in \Lambda} q ^ {x_1 \cdot x_1 + x_2 \cdot x_2}
		= \sum_{x_1 \in \Omega} q ^ {x_1 \cdot x_1} \cdot 
		\sum_{x_2 \in \Omega} q ^ {x_2 \cdot x_2}
	$$
\end{proof}

\begin{claim}
	$$\theta_{\ZZ^n}=\theta_{\ZZ}^n = \theta_{00}^n(0, \tau)$$
\end{claim}
\begin{proof}
	
\end{proof}



\begin{definition}
	Усредненный тета-ряд решетки $\Lambda$ склеенной по векторам $u_l, l=1...s$
	$$
		\Game = \cup_{l=0}^s (u_l + \Lambda)
	$$
	$$
		\theta_{\Game}(q)=\frac{1}{s} \sum_{l=1}^s \sum_{k=1}^s 
			\sum_{x \in \Lambda} q^{N(x + u_l - u_k)}
		= \theta_{\Lambda}(q) + \frac{2}{s} \sum_{l < k} 
			\sum_{x \in \Lambda} q^{N(x + u_l - u_k)}
	$$
\end{definition}

\begin{definition}
	$D_n = \{x \in \ZZ^n | \sum_i x_i = 0 (\mod 2)\}$
\end{definition}


\begin{lemma}
	Тета-ряд решетки $D_n$ равен 
	$$
		\frac{1}{2} \cdot (\theta_3^n(q) + \theta_4^n(q))
	$$
\end{lemma}

\begin{proof}
	Сначала индукцией по $n$ покажем, что если $a_k$ -- коэффициент при $q^k$ у ряда $\theta_{3}^n(q)$, то у ряда $\theta_{4}^n(q)$ коэффициент при $q^k$ равен 
	$(-1)^k a_k$.
	
	База $n=1$ сразу следует из определения. Обозначим
	 $\theta_{3}(q)=\sum_{k=0}^\infty b_k q^k$. 
	
	Шаг индукции: $\theta_{3}^{n-1}(q) = \sum_{k=0}^\infty a_k q^k,
	\quad \theta_{4}^{n-1}(q)=\sum_{k=0}^\infty (-1)^k a_k q^k$
	
	$$
		\theta_{3}^{n} = \sum_{k=0}^\infty a_k q^k \cdot \sum_{k = 0}^\infty b_k q^k = 
		\sum_{k = 0}^\infty \Big( \sum_{j=0}^k a_j b_{k-j} \Big) q^k
	$$
	$$
			\theta_{4}^{n} = \sum_{k=0}^\infty (-1)^k a_k q^k \cdot \sum_{k = 0}^\infty (-1)^k b_k q^k = 
	\sum_{k = 0}^\infty \Big(  \sum_{j=0}^k (-1)^j a_j (-1)^{k-j} b_{k-j} \Big) q^k =
	\sum_{k = 0}^\infty \Big( \sum_{j=0}^k a_j b_{k-j} \Big) (-1)^k q^k
	$$
	
	Теперь рассмотрим произвольный $x \in \ZZ^n$. Очевидно, что 
	$\sum x_i = 0 \mod 2 \iff \sum x_i^2 = 0 \mod 2$. 
	Это означает, что тета-ряд $D_n$ есть тета-ряд $\ZZ^n$ с вычтенными нечетными степенями, что в точности равно $$1/2 \cdot (\theta_3^n (q) + \theta_4 ^n (q))$$.
\end{proof}

\begin{lemma}
	Тета-ряд решетки $(1/2, ..., 1/2) + D_n$ равен 
	$
	\frac{1}{2} \cdot \theta_{2}^n (q)
	$
\end{lemma}

\begin{proof}
	Рассмотрим множество векторов $x \in D^n$ т.ч. $\sum_i ({x_i} + 1/2)^2=k$.
	Покажем, что оно биективно множеству $y \in \ZZ^n$ т.ч. $\sum_i y_i = 1 \mod 2$ и $\sum_i ({y_i} + 1/2)^2=k$.
	
	Отображение $\phi(x_1, x_2, ..., x_n) = (-x_1 - 1, x_2, ..., x_n)$ является искомой биекцией:
		$$(-x_1 - 1 + 1/2) = (-x_1 - 1/2) = (x_1 + 1/2)^2 \implies \phi(x) \cdot \phi(x) = x \cdot x = k$$
		
	Если $x \ne \xi$, то $\exists i$ т.ч. $x_i \ne \xi_i$ и тогда очевидно $\phi(x)_i \ne \phi(\xi)_i$.
	
	Пусть $y$ т.ч. $\sum_i y_i = 1 \mod 2$. Тогда взяв $x=(-y_1-1, y_2, ..., y_n) \in D_n$, получим $\phi(x)=y$.
	
	В силу конечности множества $\{x \in D_n | (x + (\frac{1}{2}^n)) \cdot (x + (\frac{1}{2}^n)) = k\}$
	мы получили, что
	
	\begin{gather*}
		\#\{x \in \ZZ^n | N(x  + (\frac{1}{2}^n)) = k\} = \\
= \#\{x \in \ZZ^n | \sum_{x_i} = 0 \mod 2,  N(x  + (\frac{1}{2}^n)) = k\} + \\
+  \#\{x \in \ZZ^n | \sum_{x_i} = 1 \mod 2,  N(x  + (\frac{1}{2}^n)) = k\} = \\
=	2 \cdot \#\{x \in D_n | N(x  + (\frac{1}{2}^n)) = k\}
	\end{gather*}

	Где в левой части стоит коэффициент перед $q^k$ в разложении $\theta_2^n(q)$ 
	(или тета-ряда $\ZZ^n$), а в правой части удвоенный коэффициент перед $q^k$ в 
	разложении тета-ряда решетки $D_n + (1/2, ..., 1/2)$, что и требовалось.
\end{proof}

\begin{theorem}
	Тета-ряд решетки 
	$$
		D_n^+ = D_n \cup (D_n + (\frac{1}{2}^n))
	$$
	равен 
	$$
		\frac{1}{2} \cdot (\theta_{2}^n(q) + \theta_{3}^n(q) + \theta_4^n(q))
	$$
\end{theorem}
\begin{proof}
	По определению, 
	$$
	\theta_{D_n^+}(q) 
	= \theta_{D_n}(q) + \sum_{x \in D_n} q^{N(x + (\frac{1}{2}^n))}
	= \theta_{D_n} + \theta_{D_n + (\frac{1}{2}^n)}
	$$
\end{proof}

\section{Тета-ряды как модулярные формы}

Покажем, относительно каких подгрупп $SL_2(\ZZ)$ тета-ряды решеток $\ZZ^{16}, \ZZ^4 \oplus D_{12}^+, D_{16}^+$, интересных с точки зрения теории суперструн, будут модулярными формами:


\begin{theorem}
	Тета-ряды решеток $\ZZ^{16},  \ZZ^4 \oplus D_{12}^+$ будут модулярными формами
	веса 8 относительно подгруппы $\Gamma_2 \subset SL_2(\ZZ)$.
	
	Тета-ряд решетки $D_{16}^+$ будет модулярной формой веса 8 относительно всей $SL_2(\ZZ)$.
\end{theorem}

\begin{proof}
	Для начала рассмотрим действие $\gamma \in \Gamma_2$ на тета-ряд решетки $\ZZ^{2}$:
	$$
		\theta_{\ZZ^2}(\gamma(\tau))=\theta_{00}^2(0, \gamma(\tau)) = \zeta^2 e_{\gamma}(\tau) \theta^2_{00}(0, \gamma(\tau))
	$$
	где $\zeta^8=1$, $e_\gamma(\tau)=c \tau + d$. Отсюда сразу вытекает требуемое равенство для тета-ряда $\ZZ^{16}$.
	
	Для решетки $D_{16}^+$ модулярность тета-ряда относительно всей $SL_2(\ZZ)$ очевидна в силу Утверждения XXX и равенства:
	$$\theta_{D_{16}^+}=\frac{1}{2}(\theta_{00}^{16}(0, \tau) + \theta_{01}^{16}(0, \tau) + \theta_{10}^{16}(0, \tau))$$
	Его вес равен 8, потому что каждый из весов 
	$\theta_{00}^2, \theta_{01}^2, \theta_{10}^2$ равен 2.
	
	Теперь займемся решеткой $\ZZ^4 \oplus D_{12}^+$. Ее тета-ряд равен $\theta_{\ZZ^4} \cdot \theta_{D_{12}^+} = \theta_{00}^4 \cdot (\theta_{00}^{12} + \theta_{01}^{12} + \theta_{10}^{12})$. Посмотрим, как $\gamma \in \Gamma_2$ действует на нем:

\begin{equation*}
	\begin{split}
		\theta_{\ZZ^4}(\gamma(\tau)) \cdot \theta_{D_{12}^+}(\gamma(\tau)) = \\		
		\zeta^4 e_{\gamma}(\tau)\theta_{00}^4 (\tau) \cdot
		(
			\zeta^{12} e_{\gamma}^3(\tau)\theta_{00}^{12} (0, \tau)
			+ \zeta^{12} e_{\gamma}^3(\tau)\theta_{01}^{12} (0, \tau)
			+ \zeta^{12} e_{\gamma}^3(\tau)\theta_{10}^{12} (0, \tau)
		)
	\end{split}
\end{equation*}
	где $\zeta^8=1$ и $e_\gamma(\tau) = (c\tau + d)^2$.
	
	Окончательно получаем:
	$$
		\theta_{\ZZ^4}(\gamma(\tau)) \cdot \theta_{D_{12}^+}(\gamma(\tau))
		=
		e_{\gamma}^4(\tau)\cdot \theta_{00}^4 \cdot (\theta_{00}^{12} + \theta_{01}^{12} + \theta_{10}^{12})
	$$
	
\end{proof}

\section{$\theta(x, it)$ как сумма дельта-функций}

Рассмотрим функцию $\theta(x, it)$, $x, t \in \RR,  t \ge 0$. Она является вещественнозначной:

$$
\theta(x, it)=
\sum_{n \in \ZZ} \exp(2 \pi i n x - \pi n^2 t)
=1 + 2\sum_{n \in \NN} \exp(-\pi n^2 \tau) \cos(2 \pi n x)
$$

Рассмотрим предел ее действие при $t \to 0$ как обобщенной функции на пространстве периодических $\RR \to \RR$:

$$f(x)=\sum_{m \in \ZZ} a_m \exp{(2 \pi i m x)}$$

\begin{equation*}
	\begin{split}
		\int_{\RR} dx \cdot \theta(x, it) f(x) = 
		\sum_{k \in \ZZ} \int_{[k, k+1]} dx \cdot \theta (x, it) f(x) = \\
		\sum_{k \in \ZZ} \int_{[0, 1]} dx \cdot \theta (x + k, it) f(x + k) =  
		\sum_{k \in \ZZ} \int_{[0, 1]} dx \cdot \theta (x, it) f(x + k) \\
	\end{split}
\end{equation*}

...

...

...

...


\section{Открытые вопросы}

\begin{enumerate}
	\item Верно ли, что любая параболическая форма веса $n \ge 3$ равна полиному
	степени $2n$ от функций $\theta_{a, b}(0, \tau)$.
	\item Можно ли записать модулярные формы $\theta_{a, b}(0, n\tau)$ в виде
	$$
	\frac{\text{квадратичный полином от} \quad \theta_{c, d}}{\text{линейная комбинация} \quad \theta_{c, d}}
	\quad ?
	$$
\end{enumerate}

\begin{thebibliography}{999}
	
	\bibitem{mumford}
	Д. Мамфорд.
	\emph{Лекции о тета-функциях}.
	
	\bibitem{conway}
	Дж. Конвей, Н. Слоэн. 
	\emph{Упаковки шаров, решетки и группы}.
	
	\bibitem{barkowsky}
	
\end{thebibliography}


\end{document}
